\documentclass[11pt, onecolumn]{article}

\usepackage{times,url}
\usepackage{multirow}
\usepackage{tabularx}
\usepackage{rotating}
\usepackage{xcolor}
\usepackage{array}

\usepackage[sort]{cite}

\makeatletter
\newcommand*{\rom}[1]{\expandafter\@slowromancap\romannumeral #1@}
\makeatother

\def\Section {\S}

\newtheorem{theorem}{Theorem}[section]
\newtheorem{observation}[theorem]{Observation}
\newtheorem{lemma}[theorem]{Lemma}
\newtheorem{proposition}[theorem]{Proposition}
\newtheorem{corollary}[theorem]{Corollary}

\newenvironment{definition}[1][Definition]{\begin{trivlist}
\item[\hskip \labelsep {\bfseries #1}]}{\end{trivlist}}
\newenvironment{example}[1][Example]{\begin{trivlist}
\item[\hskip \labelsep {\bfseries #1}]}{\end{trivlist}}
\newenvironment{remark}[1][Remark]{\begin{trivlist}
\item[\hskip \labelsep {\bfseries #1}]}{\end{trivlist}}

% Macros:
\newcommand{\ie}{{\it i.e.}}
\newcommand{\eg}{{\it e.g.}}
\newcommand{\cf}{{\it cf.}}
\newcommand{\etc}{{\it etc.}}
\newcommand{\viz}{{\it viz.}}
\newcommand{\apriori}{{\it a priori}}
\newcommand{\eat}[1]{}

% Notes:
\newcommand{\num}[1]{{\color{red}\bf {#1}}}

\newcommand{\giulia}[1]{{\color{red} {#1}~(Giulia)}}
\newcommand{\wenting}[1]{{\color{blue}{#1}~(Wenting)}}


% Signed quotes:
\def\signed #1{{\leavevmode\unskip\nobreak\hfil\penalty50\hskip2em
  \hbox{}\nobreak\hfil(#1)%
    \parfillskip=0pt \finalhyphendemerits=0 \endgraf}}

\newsavebox\mybox
\newenvironment{aquote}[1]
  {\savebox\mybox{#1}\begin{quote}}
    {\signed{\usebox\mybox}\end{quote}}

\sloppy
\begin{document}

\title{Progress Report: Deanonymizing Anonymous Messaging Social Networks}

\author{Giulia Fanti and Wenting Zheng}

   \date{}
   \maketitle
   \thispagestyle{empty}
\textbf{Sketch of project:} Our project is an investigation of the anonymity properties of commercial anonymous social networks, such as Secret, Whisper, and Yik Yak. These networks allow people to spread messages over a contact network without authorship information attached. The spreading mechanism occurs as follows: each time a user approves a message (e.g., “retweeting  on Twitter), that message becomes visible to the user's neighbors on the social graph. 
%
We are considering an adversary who wants to learn the authors of messages; this implies the adversary does not have access to the service provider's internal data stores. An example adversary would be a government agency that wants to secretly de-anonymize identities in the social network by compromising nodes or implanting spies within the network.
%The adversary consists of colluding “spy” nodes in a social network. This is meant to simulate a government agency that recruits agents to participate in a social network in order to monitor the content that flows through it. 
Our hypothesis is that such an adversary can learn a lot of information about the true source of the message simply by an offline analysis of timestamped messages intercepted by these compromised nodes.

To test our hypothesis, we simulate the spread of messages over various networks in the presence of \emph{spy nodes}. Given a simulated message spread, we try to infer the true source from the spy nodes’ intercepted messages.  We vary attributes such as graph topology and size, the number and distribution of spy nodes, the probability of a node approving a message. We also utilize different estimation algorithms to attempt to infer the true source of the message.
Our goal is to understand the adjustable parameters' impact on the efficiency of an attack -- for example, how many spies an adversary must corrupt in order to reliably infer the source of a message, regardless of where in the network it originates.

\vspace{0.1in}
\textbf{Work completed:} Thus far, we have written a simulator for spreading messages over a connected network. This simulator can generate synthetic graphs (Barabasi-Albert, Watts-Strogatz), and it can also use existing graph datasets. We have found a Facebook dataset that we plan to use in our simulation, which contains the Facebook links among about 10,000 nodes in 2009. Given a graph, the simulator allows nodes to make decisions about when and how they will spread a message to their neighbors. 
The simulator is implemented as a discrete-time system. Nodes decide to approve messages by drawing samples from a Bernoulli random variable with parameter $0.5$. The network latency is modeled with a geometric random variable with parameter $p$. 
At the start of a simulation run, a certain percentage of the nodes are compromised. We are currently compromising \emph{random} nodes in the network, though we hope to explore different layout of spy nodes. 
At the end of a simulation run, the list of timestamped messages received by malicious nodes are stored in a file and input into estimation algorithms.

We have also implemented a Jordan centrality estimator, which guesses that the true message source is the node with the lowest Jordan centrality. The Jordan centrality of a node is defined as the distance from that node to the farthest node on the graph. In our case, we are finding the node with the minimum distance to the farthest-away spy node in the network. This is the most basic estimator we will test, as it does not even use timing information from the spies. Note that if we were to use this estimator as time tends to infinity, the estimate would always be the same---namely, the most ``central" node in the graph. So we will start by using this estimator at different points in time to infer how the estimate changes over time. We expect the best estimates to occur when the timestamp is small.

Thus far, we have used our simulator to simulate message spreads over Barabasi-Albert graph. However, we have not run our estimation algorithms yet, so we do not know how well the adversary can infer the true message source.

Additionally, we have changed our project definition slightly after realizing that inserting Sybils into the network is equivalent to saying that the legitimate end of Sybil attack edges are spies. Therefore, Sybils do not give any additional information; as such, we can understand the threat that Sybils pose by simply quantifying the proportion of network nodes that befriend Sybils, and treating them as spies. 

\vspace{0.1in}
\textbf{What remains:} At this point, the number 1 priority is developing a more sensitive and methodical estimator than the Jordan estimator. In particular, we need to develop a method for incorporating timing information in our source estimates. 

Once that is complete, we will run different simulations in order to explore the parameter space adequately. We will also consider more complex estimators, but these are likely to be heuristics, since optimal estimators in this space are difficult to compute. It seems like the path forward will be fairly methodical.

We are also planning to test estimation when the adversary knows only a subset of the underlying graph. This is a more realistic scenario in practice, because it might be difficult for an adversary to learn the whole graph if the social network is maintained by a private company. Of course, this depends on the strength of the adversary. 
We plan to sample the underlying real graph to create a subgraph.
In this scenario, we may need to develop new heuristic methods for inferring the true message source over partial estimates of the underlying network based on the simulation results.
%Then we will need to run the simulations from the full-graph scenario on the partial graph. 

\vspace{0.1in}
\textbf{Concerns/open issues:}
Our main concerns at this point are: 1) The heuristic estimators we come up with will not detect the source adequately, and 2) the computational load of these simulations will become too prohibitive to explore the parameter space fully. We can try to optimize the code somewhat in this regard, but parallelizing graph algorithms may be difficult. 
Regarding the first concern, we are fairly confident that inference is possible, but finding a computationally efficient (and strong) estimator may be difficult. One option we are considering is assuming that the message traverses the shortest path from the source to each spy node, and that each path is independent. This is not a good assumption, as the message's path to nearby spies is likely to be highly correlated, but it might still give good estimates.

\vspace{0.1in}
\textbf{Need for meeting/availability:} We don't think we need a meeting (unless you feel otherwise).

\vspace{0.1in}
\textbf{Presentation slot preference:} We would prefer to present on May 6 or 8, as Giulia has to proctor a midterm on May 1 and may need to proctor during our usual class slot.
%\bibliography{references}
%\bibliographystyle{ieeetr}
\end{document}
