\section{Discussion}

The results in \S\ref{sec:eval} reveals some potential problems for the maximum likelihood estimator. 

There are various aspects of ML estimator that do not make sense algorithmically. 

\begin{itemize}
\item The estimator gives a log likelihood of 0 if a node $s$ is equidistant from all spy nodes, even if the timing information shows that $s$ is unlikely to be the true source. For example in Fig.~\ref{??}, we see that the ~~~~
\end{itemize}

The estimator's pruning process is questionable. Given a graph $G$, the estimator prunes the graph using the direction of the propagation message. For every spy node $n_s$ that receives a message, the node prunes all edges that are not the propagation edge. After the pruning is done, the estimator discards all disconnected components except for the one containing the spy node with the earliest delay. After the pruning is done, the maximum likelihood estimator is run on the graph.

This is pruning process is similar to the first-spy estimator, which picks a random neighbor that is closest to the 
