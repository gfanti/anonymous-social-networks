\begin{abstract}
Over the last year, anonymous microblogging applications like Yik Yak, Whisper, and Secret gained popularity. These services spread messages to  contacts without including any authorship information. In this project, we wish to explore whether a moderately powerful third-party adversary, e.g., a government agency or a school administration, could deanonymize messages by simply recruiting network participants as spies and knowing the underlying social graph.

We initially hypothesized this might be possible due to a recent body of work on locating the patient-zero for a disease spreading over a graph. In our case, the disease corresponds to a message, and infection refers to nodes passing along the message to their contacts in the anonymous social network. Existing work on infection source detection assumes that the estimator has strong side information, including the underlying graph structure and information about who transmitted the disease/message to whom. In this work, we will mainly examine the latter assumption. 
We will discuss our experiences trying to replicate the results presented in the literature and apply these estimators to social networks. %We have found deanonymization to be significantly more challenging than we previously believed, and our project has been largely been spent trying to understand why that is.
Our results suggest that when spy nodes do not have information about the direction of infection, deanonymization accuracy suffers, but the adversary can still achieve moderately high deanonymization rates.

%% Over the last year, anonymous microblogging applications like Yik Yak, Whisper, and Secret gained popularity. These services spread messages to contacts without including any authorship information. In this project, we wish to explore whether a moderately powerful third-party adversary, e.g., a government agency or a school administration, could deanonymize messages by simply participating in the network and knowing the underlying social graph.

%% We believed this might be possible due to a recent body of work on locating the patient-zero for a disease spreading over a graph. In our case, the ``disease'' corresponds to a message, and nodes are ``infected'' once they observe the message. Existing work assumes that the estimator can uniformly sample the graph to learn a subset of the infected nodes; given this, there exist strong guarantees and tractable methods for inferring an infection source.

%% We will discuss our experiences trying to replicate the results presented in the literature and apply these estimators to social networks. We have found deanonymization to be significantly more challenging than we previously believed, and our project has been largely been spent trying to understand why that is.

\end{abstract}
