\section{Introduction}

Recently, there has been a rise in the number of anonymous social networks. Such social networks, including Secret \cite{secret}, Whisper \cite{whisper}, and Yik Yak \cite{yikyak}, promise users that they can post messages anonymously; that is, it will be impossible to tell who sent what message. Under such promises, anonymous social network users may be tempted to post sensitive information. For example, a user may post messages disapproving of a political candidate running for office. In oppressive regimes, the candidate's party may then have incentive to track down the person who actually posted the message. In this project, we ask \textbf{how easy is it for a \emph{moderately} powerful adversary to track down the true source of the message?}

We do not assume a state-level adversary, such as a national government or the NSA, as it would be easy for them to deanonymize these social networks. Services like Secret use a centralized server to propagate users' messages. Powerful adversaries like government agencies would have the technology to break into those centralized servers themselves, or have the power to utilize legal measures to request message sources directly. Instead, we consider an adversary that is only \emph{moderately} powerful, such as a school administrator or a dissenting political opponent. These adversaries do not have access to the centralized server, but they do have access to certain \emph{spies} within the social network. These spies listen passively for messages and intercept them. After a certain time has elapsed, the spies come together and use the collected information to find the true source of the propagation. The intercepted messages also contain the receiving timestamps, which are crucial for tracking down the source.

In this project, we investigate the posed question by implementing multiple message source detection algorithms and testing these algorithms in a simulation setting on various types of graphs. We take ideas from research in other domains studying infection sources (i.e. identifying the patient zero of an infectious disease outbreak in a region) and apply those algorithms to the anonymous social network setting. 


More specifically, we would like to answer two questions:
\begin{enumerate}
\item How many spies within the network must an adversary have in order to effectively find the true source of the message?
\item How should an adversary select spies in order to maximize the probability of finding the sources of messages? 
\end{enumerate}
%We believe we can show through simulation that networks are susceptible to deanonymization by adversaries with limited power to create friendships and/or corrupt nodes.
%We simulate the spread of messages over various networks in the presence of \emph{spy nodes}. Given a simulated message spread, we try to infer the true source from the spy nodes’ intercepted messages.  
We initially intended to vary attributes such as graph topology and size, the number and distribution of spy nodes, and the probability of a node approving a message. Due to some significant roadblocks, we ended up only studying a subset of these questions. We also utilize different estimation algorithms to attempt to infer the true source of the message. 

