\documentclass[10pt,preprint,nocopyrightspace]{sigplanconf}
%
%% SOME OF OUR DEFINITIONS
%
\usepackage{xspace}
%\newcommand{\sys}{MiniCrypt\xspace}


\usepackage{fancyvrb}
\usepackage{listings}

\usepackage[compact]{titlesec}
%\titleformat{\section}{\normalfont\bfseries}{\thesection}{1em}{}
%\titlespacing*{\section}{1pt}{*0}{1pt}

\newcommand{\longerpaper}[1]{}

\usepackage{numprint}

\usepackage{multirow}

\lstset{
    basicstyle=\footnotesize\ttfamily,
      columns=fullflexible,
        keepspaces=true,
        }

\usepackage{subfigure}
\usepackage{ifpdf}
\usepackage[usenames,dvipsnames]{color}
\usepackage[letterpaper, breaklinks, pdfborder={0 0 0}]{hyperref}
\hypersetup{
    backref=true
      bookmarksnumbered,
        colorlinks=true,
          pdfstartview={FitH},
            citecolor={blue},
              linkcolor={blue},
                urlcolor={blue},
                  citecolor={blue},
                    pdfpagemode={UseOutlines}
                      }

  %\usepackage{algorithmic}

\usepackage[noend]{algpseudocode}

% edit the comments
\usepackage{eqparbox}
\renewcommand{\algorithmiccomment}[1]{\eqparbox{COMMENT}{// {\emph
      #1}}}

%\usepackage{multicol}
\usepackage{amsmath, amssymb}
\usepackage{amsthm}
\usepackage{enumitem}
%\usepackage{rotating}
%\onehalfspacing
%\newcommand{\tbd}[1]{[{\bf{#1}}]}
\newcommand{\tbd}[1]{}
\newcommand{\ie}{{\it i.e.}}
\newcommand{\eg}{{\it e.g.}}
\newcommand{\etc}{{\it etc.}}
\newcommand{\eat}[1]{}


\newcommand{\mypara}[1]{\noindent{\bf {#1}.}~}
\newcommand{\submypara}[1]{\medskip\noindent{\it {#1}.}~}
\newcommand{\chk}{$\checkmark$}
\newcommand{\dsh}{{\bf --}}
\newcommand{\til}{{\bf\large \textasciitilde}}
\usepackage{rotating}

\usepackage{framed}


% for packing
\newcommand{\mem}{\mathsf{MemorySize}\xspace}
\newcommand{\data}{\mathsf{Data}\xspace}
\newcommand{\cratio}{\mathsf{cr}\xspace}
\newcommand{\maxcr}{\mathsf{maxcr}\xspace}
\newcommand{\sizeof}{\mathsf{sizeof}\xspace}
\newcommand{\nmax}{n_{\max}}
\newcommand{\rowsize}{\mathsf{rowsize}}
\newcommand{\diskbw}{\mathsf{DiskTransferBandwidth}\xspace}
\newcommand{\diskseek}{\mathsf{DiskSeekTime}\xspace}


% RESULTS


\usepackage{enumitem}
\setlist[itemize]{itemsep=0.00cm}
\setlist[itemize]{parsep=0.01cm}
%\setlist[itemize]{parskip=0.01cm}


\newenvironment{myitemize}
{ \begin{itemize}[nolistsep]
        \setlength{\itemsep}{0.5pt}
            \setlength{\parskip}{0.9pt}
                \setlength{\parsep}{0.1pt}     }
{ \end{itemize}                  }

\newcommand{\green}[1]{{\color{ForestGreen}{#1}}}


% our encryption scheme

\newcommand{\keygen}{\mathsf{KeyGen}}
\newcommand{\Enc}{\mathsf{Enc}}
\newcommand{\enc}{\Enc}
\newcommand{\en}{\mathsf{enc}}


\renewcommand{\mod}{\mathsf{mod}\xspace}

% requires amsthm, enumitem
%\theoremstyle{algorithms}
%\newtheorem{construction}{Algorithm}
%\newcommand{\ALGORITHM}[4]{%  name, llabel, intro, \items
%\begin{construction}[#1]\label{#2} \mbox{}
%\normalfont\noindent
% #3
% \vspace{0.13\baselineskip}
% \begin{enumerate}[noitemsep,nolistsep]\itemsep=0.1\baselineskip
% #4
% \end{enumerate}
% \end{construction}
%}



\newcommand{\ALGORITHM}[4]{%  name, llabel, intro, \items
  \let\oldi\labelenumi
  \let\oldii\labelenumii
  \let\oldiii\labelenumiii
  \renewcommand{\labelenumi}{\arabic{enumi}: }
  \renewcommand{\labelenumii}{\arabic{enumi}.\arabic{enumii}: }
  \renewcommand{\labelenumiii}{\arabic{enumi}.\arabic{enumii}.\arabic{enumiii}:
  }
  \noindent \textbf{#1:} \label{#2}
  #3
   \vspace{0.13\baselineskip}
    \begin{enumerate}[noitemsep,nolistsep]\itemsep=0.1\baselineskip
       #4
       \end{enumerate}
     \let\labelenumi\oldi
     \let\labelenumii\oldii
     \let\labelenumiii\oldii
     }

%\newcommand{\subparagraph}{\paragraph}
%\usepackage[compact]{titlesec}

%\newcommand{\simplequote}[1]{\begin{quote}{\it{#1}}\end{quote}}

%\renewcommand\%bibname{}


%\usepackage[compact]{titlesec}
%\titlespacing{\section}{0pt}{0pt}{0pt}
%\titlespacing*{?command?}{?left?}{?beforesep?}{?aftersep?}[?right?]



\newcommand{\tickYes}{\checkmark}

% people
\newcommand{\ion}[1]{{\color{Red} IS: {#1}}}
\newcommand{\rachit}[1]{{\color{Red} RAg: {#1}}}
\newcommand{\rp}[1]{{\color{Red} RPo: {#1}}}
\newcommand{\frank}[1]{{\color{CarnationPink} FL: {#1}}} % :)
\newcommand{\wz}[1]{{\color{SkyBlue} WZ: {#1}}} % :)
\newcommand{\qu}[1]{{\color{Magenta}  {\bf Question:} {#1}}}
\newcommand{\warning}[1]{{\color{Red}{\bf Warning: #1}}}
\newcommand{\todo}[1]{{\color{Red} [todo: {#1}]}}

\newcommand{\aes}{\mathsf{AES}}
\newcommand{\encr}{\mathsf{enc\_r}}
%
\newcommand{\get}{\texttt{get}}
\newcommand{\delete}{\texttt{delete}}
\renewcommand{\put}{\texttt{put}\@\xspace}
\newcommand{\key}{\texttt{key}\xspace}
\newcommand{\val}{\texttt{value}\xspace}
\newcommand{\ver}{\texttt{version}\xspace}
\newcommand{\packID}{\texttt{packID}\xspace}
\newcommand{\pvalue}{\texttt{pvalue}\xspace}
\newcommand{\low}{\texttt{low}\xspace}
\newcommand{\high}{\texttt{high}}
\newcommand{\merge}{\texttt{merge}}
\newcommand{\msplit}{\texttt{split}}
\newcommand{\result}{\texttt{result}}
\newcommand{\pack}{\texttt{pack}\xspace}
\newcommand{\ts}{\mathsf{ts}}


% Compact itemize and enumerate.  Note that they use the same counters and                         
% symbols as the usual itemize and enumerate environments.                                         
\makeatletter
\def\compactify{\itemsep=3pt plus3pt \topsep=3pt plus3pt \partopsep=0pt
\parsep=0pt \leftmargin=1.3em}
\let\latexusecounter=\usecounter
\def\CompactItemize{%                                                                              
  \ifnum \@itemdepth >\thr@@\@toodeep\else
    \advance\@itemdepth\@ne
    \edef\@itemitem{labelitem\romannumeral\the\@itemdepth}%                                        
    \expandafter
    \list
      \csname\@itemitem\endcsname
      {\compactify\def\makelabel##1{\hss\llap{##1}}}%                                              
  \fi}
\let\endCompactItemize\endlist
\newenvironment{CompactEnumerate}
  {\def\usecounter{\compactify\latexusecounter}
   \begin{enumerate}}
  {\end{enumerate}\let\usecounter=\latexusecounter}
\makeatother



\begin{document}



%
% --- Author Metadata here ---
%\conferenceinfo{WOODSTOCK}{'97 El Paso, Texas USA}
%\CopyrightYear{2007} % Allows default copyright year (20XX) to be
%over-ridden - IF NEED BE.
%\crdata{0-12345-67-8/90/01}  % Allows default copyright data
%(0-89791-88-6/97/05) to be over-ridden - IF NEED BE.
% --- End of Author Metadata ---

\title{A Study on Deanonymizing Anonymous Social Networks}

% other title: or An encrypted and compressed key-value store \\
% Reconciling encryption and compression for key-value stores
% other paper names:  XZip, EncZip, Enczipt, CryptZip, CZip, EZip,
% Cryptar/CrypTar  (crypt+tar), Tarx, Tarc (tar +crypt), MiniCrypt



%
% You need the command \numberofauthors to handle the 'placement
% and alignment' of the authors beneath the title.
%
% For aesthetic reasons, we recommend 'three authors at a time'
% i.e. three 'name/affiliation blocks' be placed beneath the title.
%
% NOTE: You are NOT restricted in how many 'rows' of
% ``name/affiliations'' may appear. We just ask that you restrict
% the number of 'columns' to three.
%
% Because of the available 'opening page real-estate'
% we ask you to refrain from putting more than six authors
% (two rows with three columns) beneath the article title.
% More than six makes the first-page appear very cluttered indeed.
%
% Use the \alignauthor commands to handle the names
% and affiliations for an 'aesthetic maximum' of six authors.
% Add names, affiliations, addresses for
% the seventh etc. author(s) as the argument for the
% \additionalauthors command.
% These 'additional authors' will be output/set for you
% without further effort on your part as the last section in
% the body of your article BEFORE References or any Appendices.

%\numberofauthors{5} %  in this sample file, there are a *total*
% of EIGHT authors. SIX appear on the 'first-page' (for formatting
% reasons) and the remaining two appear in the \additionalauthors
% section.
%
\authorinfo{}

%\date{30 July 1999}
% Just remember to make sure that the TOTAL number of authors
% is the number that will appear on the first page PLUS the
% number that will appear in the \additionalauthors section.

\maketitle



\begin{abstract}
Over the last year, anonymous microblogging applications like Yik Yak, Whisper, and Secret gained popularity. These services spread messages to  contacts without including any authorship information. In this project, we wish to explore whether a moderately powerful third-party adversary, e.g., a government agency or a school administration, could deanonymize messages by simply recruiting network participants as spies and knowing the underlying social graph.

We initially hypothesized this might be possible due to a recent body of work on locating the patient-zero for a disease spreading over a graph. In our case, the disease corresponds to a message, and infection refers to nodes passing along the message to their contacts in the anonymous social network. Existing work on infection source detection assumes that the estimator has strong side information, including the underlying graph structure and information about who transmitted the disease/message to whom. In this work, we will mainly examine the latter assumption. 
We will discuss our experiences trying to replicate the results presented in the literature and apply these estimators to social networks. %We have found deanonymization to be significantly more challenging than we previously believed, and our project has been largely been spent trying to understand why that is.
Our results suggest that when spy nodes do not have information about the direction of infection, deanonymization accuracy suffers, but the adversary can still achieve moderately high deanonymization rates.

%% Over the last year, anonymous microblogging applications like Yik Yak, Whisper, and Secret gained popularity. These services spread messages to contacts without including any authorship information. In this project, we wish to explore whether a moderately powerful third-party adversary, e.g., a government agency or a school administration, could deanonymize messages by simply participating in the network and knowing the underlying social graph.

%% We believed this might be possible due to a recent body of work on locating the patient-zero for a disease spreading over a graph. In our case, the ``disease'' corresponds to a message, and nodes are ``infected'' once they observe the message. Existing work assumes that the estimator can uniformly sample the graph to learn a subset of the infected nodes; given this, there exist strong guarantees and tractable methods for inferring an infection source.

%% We will discuss our experiences trying to replicate the results presented in the literature and apply these estimators to social networks. We have found deanonymization to be significantly more challenging than we previously believed, and our project has been largely been spent trying to understand why that is.

\end{abstract}

\section{Introduction}

Recently, there has been a rise in the number of anonymous social networks. Such social networks \todo{cite here} often promise users that they are able to post messages such that it is impossible to tell who sent what message. The propagated messages do not have 
Under such promises, anonymous social network users may be tempted to post relatively sensitive information. For example, a user may post messages disapproving a political candidate running for office. The candidate will then have incentive to track down the person who actually posted the message. In this project, we want to ask the question: \textbf{how easy is it for a \emph{moderately} powerful adversary to track down the true source of the message?}

In this project, we investigate the posed question by implementing multiple message source detection algorithms and testing these algorithms in a simulation setting on various types of graphs. We take ideas from other domains studying infection sources (i.e. identifying the patient zero of an infectious disease outbreak in a region) and apply those algorithms to the anonymous social network setting. 

We do not assume a very powerful adversary such as the NSA. Services like Secret actually utilize a centralized server to propagate the users' messages, and government agencies would have the power to either break into those centralized servers themselves, or have court orders to request for message sources directly. Instead, we consider an adversary who is only \emph{moderately} powerful, such as a school administrator or a dissenting political opponent. These adversaries do not have access to the centralized server, but they do have access to certain \emph{spies} within the social network. These spies listen passively for messages and intercept them. After a certain time has elapsed, the spies come together and use the collected information to find the true source of the propagation. The intercepted messages also contain the receiving timestamps, which are crucial for tracking down the source.

More specifically, we would like to answer two questions
\begin{enumerate}
\item How many spies within the network must an adversary have in order to effectively find the true source of the message?
\item How should an adversary select spies in order to maximize the probability of finding the sources of messages? 
\end{enumerate}

%We believe we can show through simulation that networks are susceptible to deanonymization by adversaries with limited power to create friendships and/or corrupt nodes.
We simulate the spread of messages over various networks in the presence of \emph{spy nodes}. Given a simulated message spread, we try to infer the true source from the spy nodes’ intercepted messages.  We vary attributes such as graph topology and size, the number and distribution of spy nodes, the probability of a node approving a message. We also utilize different estimation algorithms to attempt to infer the true source of the message. 


\input{related}
\section{Methodology}

\subsection{Model}

We assume an application model that is similar to Secret~\cite{secret}.
\begin{figure}
\centering
\includegraphics[height = 2.4in]{figures/secret_infrastructure}
\caption{An overview of the infrastructure for a typical anonymous social network.}
\label{fig:secret_infrastructure}
\end{figure}
Fig.~\ref{fig:secret_infrastructure} shows how a typical anonymous microblogging network works. Users are connected to friends as in a regular social network; we model this social network as a graph $\mathcal G(V,E)$, where $V$ denotes the set of vertices, or participants, in the network, and $E$ denotes the set of edges. When a source node, Alice, decides to send a message to her friends, her message is routed to a centralized server first. We denote the source node by $v^*$. The server propagates the message to Alice's neighbors, $\mathcal N(v^*)$. Alice's neighbors do not know who sent or authored the message. %To them, the message is from the centralized server instead of directly from Alice, and does not have any information about the sender herself. 
If one of Alice's friends decides to ``like'' the message, the message will be further propagated to her friend's friends. Thus, whenever a user receives a message, he/she does not know if a neighboring friend sent it, or if it was simply liked by a friend. We assume that propagation occurs with some time delay, which captures the time between the server pushing a message and the receipient actually seeing it. We model the delay between a user $i$ liking a message and the user's friend $j$ seeing the message as a random variable $\theta_{ij}$; the $\theta_{ij}$'s, or delays across each edge in the social graph, are assumed to be iid random variables. We do not specify the distribution of $\theta_{ij}$ \emph{a priori}; in principle, this should be informed by measurements of real social network usage.

Our adversarial model considers a moderately powerful adversary that is able to compromise certain users in the social network. This might occur by offering incentives to users and recruiting them to act as spies. The spies are not active, but are instead passive observers. The only thing that differentiates spies from honest users is that spies collect all observed messages (along with timestamps), and then send this information to the adversary. The adversary uses the collected message timetamps to attempt to track down the perpetrator of the message.
We assume there are $K$ spies in the network, $o_1,\ldots,o_K$. For a given message $m$, each spy $o_i$ will collect the timestamp $t_i^{(m)}$ at which it first receives the message. The adversary must then output an estimate $\hat v$, which it believes to be the true source of message $m$. %Therefore, the 

The central adversary will receive different sets of messages, and it will need to group them into a \emph{collection} -- a group of messages received at different nodes that came from the same source. We assume that these messages are not encrypted, and thus the adversary can use the plaintext information directly to identify the correct collection membership. 

\subsection{Efficacy of deanonymization}

We study the ability of this adversary to deanonymize users through simulation. A number of factors could impact deanonymization:

\begin{description}
\vspace{-0.1in}
\item[Underlying graph structure:] The underlying graph structure can significantly affect deanonymization. For example, a tree-like structure with no loops is easier to deanonymize because there are fewer possible paths between nodes to analyze. In our simulation, we study a range of graph structures, including trees, Erdos-Renyi, and Barabasi-Albert graphs.
\item[Fraction of spies:] The fraction of spies will greatly impact the deanonymization performance. Fewer spies means less information, and thus makes the true source much harder to track. Using simulation, we want to \emph{quantify} how many spies are needed to efficiently find the true source.
\item[Location of spies:] We hypothesize that selecting spies as high-degree nodes will increase the probability of detection, as popular nodes are more likely to see and propagate the message. However, this may be unrealistic in practice, as popular individuals (i.e., high-degree nodes) may hold more social leverage and may therefore resist recruitment attempts. A uniform distribution of spy nodes seems feasible for an adversary with moderate but limited resources. 
\item[Propagation latency:] The propagation latencies $\theta_{ij}$ are likely to depend on many factors, such as the popularity of the social network, the habits of its users, and the application platform (e.g., mobile vs. desktop).
In simulation, we model the propagation latency in two ways: using a Gaussian distribution with a high mean-to-standard-deviation ratio, and a geometric random variable. 
\item[Estimator:] To optimize probability of detection, i.e. $P(\hat v=v^*)$, the adversary would like to use a maximum-likelihood (ML) estimator. However, ML estimators are not known for general graph structures and propagation delay models. We therefore try different methods of estimation. The majority of our effort goes towards adapting the observer model~\cite{pinto} to our spy nodes setting.
\end{description}

\subsection{Estimators}
Initially, we attempted to come up with some estimators on our own. \todo{Should we mention the Jordan estimators here?}



\subsubsection{ML estimator for trees}
The observer model estimator~\cite{pinto} is a maximum likelihood estimator. The setup is quite similar to our simulation: given a graph $\mathcal G$, there exists a set of $K$ observers that observe information transmitted in the network. The collected information has a timestamp, as well as direction information (e.g. observer $o$ receives the message from neighbor $v$). This is the main difference between \cite{pinto} and our work; we do not assume that spies have direction information. \cite{pinto} also assumes that propagation delays are modeled by iid Gaussian random variables with distribution $\mathcal N(\mu,\sigma^2)$. We start by defining some terms:

Let $\boldsymbol{d}$ be the timestamp differences of observed arrivals with respect to some reference spy node, defined as
\begin{equation}
  \boldsymbol{d}_k = t_{k} - t_1
\end{equation}
where $t_i$ denotes the observed timestamp at spy $i$, and $k = 2,\ldots, K$.

The next term is the expected delay, which is the delay that one \emph{should} observe if the information had propagated from some candidate source $s$ ($s$ is an honest node in the graph):
\begin{equation}
  \boldsymbol{\mu}_{s,k} = \mu (|P(s, o_{k+1})| - |P(s, o_1)|)
\end{equation}
where $|P(u, v)|$ denotes the number of edges on the path that connects nodes $u$ and $v$. Finally, we define a scaled covariance matrix, which represents the covariance of the jointly-gaussian delay random variable:
\begin{equation}
  \boldsymbol{\Lambda}_{k, i} = \begin{cases}
    |P(o_1, o_{k+1})| & \text{if $k = i$} \\
    |P(o_1, o_{k+1}) \cap P(o_1, o_{i+1})| & \text{if $k \neq i$}.
  \end{cases}
\end{equation}
The authors use these terms to construct a source estimator for general trees:
\begin{equation}
\label{eq:general}
\hat{s} = \arg\max_{s \in \mathcal T_{a}} \dfrac{\exp(-\frac{1}{2} (\boldsymbol{d} - \boldsymbol{\mu}_{s})^{T} \boldsymbol{\Lambda}_s^{-1} (\boldsymbol{d} - \boldsymbol{\mu}_s) }{|\boldsymbol{\Lambda}_s|^{1/2}}.
\end{equation}

$\mathcal T_a$ denotes the \emph{pruned} tree. Pruning refers to the process of removing nodes that could not have possibly generated the observed information. For instance, consider Figure \ref{fig:pruning}; the spies $o_1$ and $o_2$ both observe that they received the message from node 3. Therefore, node 1 could not possibly have been the source, and we can prune it from the tree. In this example, the pruned tree $\mathcal T_a$ consists solely of node 3. For trees, this pruning process does not change the probability of detection; it only speeds up estimation. However, we will show later that for loopy graphs, pruning significantly increases the probability of detection. In scenarios where direction information is not available (as in our application), the estimator in \cite{pinto} achieves much lower accuracies.

\begin{figure}[h]
\centering
\includegraphics[width = 3.2in]{figures/pruning}
\caption{Directionality information enables the adversary to prune the graph of nodes that could have plausibly generated the spies' observations. In this example, the pruned tree $\mathcal T_a$ consists only of node 3.}
\label{fig:pruning}
%\vspace*{-0.4in}
\end{figure}


Equation \ref{eq:general} can be applied to general graphs as a heuristic estimator. However, if the underlying graph is tree-structured, equation \ref{eq:general} simplifies to
\begin{equation}
\label{eq:tree}
\hat{s} = \arg\max_{s \in \tau_{a}} \boldsymbol{\mu}_{s}^{T} \boldsymbol{\Lambda}^{-1} (\boldsymbol{d} - \frac{1}{2}\boldsymbol{\mu}_s).
\end{equation}
because $\Lambda_s$ is constant for all candidate nodes. This estimator is provably optimal for tree-structured graphs.

\subsubsection{Nearest-spy estimator}
We also tested a heuristic estimator, which does not make use of the full available information. This heuristic estimator uses only the first spy to receive the message, i.e., the spy with the smallest timestamp. Without loss of generality, assume that spy is $o_1$. The estimator chooses $\hat s$ as a uniformly-selected honest neighbor of $o_1$. This estimator does not use direction information to prune candidates. A trivial extension would be to use direction information and pick $\hat s$ as the node that delivered the message to the first spy. We use this nearest-spy estimator with pruning to compare our results with those of \cite{pinto}, which make use of direction information. 
\section{Evaluation}
\label{sec:eval}

The evaluation of our work is executed on a discrete-event simulator written in Python. This simulator is similar to existing discrete-event simulators such as ns3. Events are put onto a priority queue ordered by time, and executed in a single loop.

This simulator takes in as input different network structures with a list of nodes, links, and node types. 
Nodes can be either honest or malicious. As mentioned before, we simulate passive adversaries: malicious nodes are honest-but-curious. Whenever a malicious node receives a message, it adds the message, along with the time at which the message is received, to a list of intercepted messages. The timestamp used is the global simulator time, thus we assume that the spies' clocks are roughly synchronized (or that the synchronization time is much smaller than). 

At the start of a simulation run, a certain percentage of the nodes are compromised. The percentage number can be tweaked to adjust the number of malicious nodes in the network. The simulator then runs for some number of time steps and produces a list of messages intercepted from malicious nodes. After the simulation, the estimators take in the list of timestamped messages and produce guesses for the true source of the message.

\subsection{Estimator validation}
We started by attempting to replicate the estimator results in \cite{pinto}. 
In this replication process, we encountered four primary mistakes or omissions in \cite{pinto} that affected our estimation accuracy; some of these were easy to correct once we identified them, others were not. All equation numbers in this list are with respect to \cite{pinto} and the corresponding supplementary materials.
\begin{enumerate}
\item There is a typo in equation (2), which describes how to compute the observed delay vector. It should say $[\boldsymbol d]_k=t_{k+1}-t_1$ instead of $[\boldsymbol d]_k=t_{k+1}-t_k$. This is a small but important typo in the description of the estimator that leads to incorrect likelihoods. 
\item Algorithm 2 of the supplemental materials describes how to compute likelihoods for general graphs. In step 6, it should say ``compute the source likelihood using equation (4) for node $s$.", rather than ``using equation (7)." This is because over general graphs, the covariance matrix $\Lambda_s$ is not identical for all nodes, which causes the simplifications in equation (7) to be invalid. As such, the likelihoods should be computed using the more general equation (4). This error initially led to incorrect likelihood computations in our code.
\item Algorithm 2 of the supplemental materials does not explain how to prune graphs that are not tree-structured. Additionally, it does not describe how to incorporate the direction of infection into estimation. These omissions collectively have a significant impact the likelihoods obtained by the algorithm, and we suspect this is partially responsible for our inability to exactly reproduce their results.
\item The parameter specifications of the random graphs in Table 1 are not given. Specifically, the Barabasi-Albert parameter is not given. We therefore tried a range of different parameters.
\end{enumerate}
For point 3, the corresponding author of \cite{pinto} sent us part of his simulation code two days before the project due date, so we have revised our discussion to reflect what we learned from reading his code. Namely, his code helped explain how they prune graphs that are not tree-structured. 

For general graphs, pruning takes place in two steps: 1) For each spy node, remove all graph edges that were not used to transmit the message to the spy. 2) Step 1 will leave the graph disconnected; keep the connected component that contains the spy with the earliest timestamp. The resultant subgraph is called $\mathcal G_a$, and the optimization in equation \ref{eq:general} occurs as normal over the honest nodes of $\mathcal G_a$ (rather than $\mathcal T_a$). 

On a tree, this pruning procedure corresponds to only using timestamps from the nearest spies; if two spies lie on the same path (as defined by the direction of message transmission), then the latter spy's information will be discarded. For example, in Figure \ref{fig:pruning}, $o_2$'s timestamp $t_2$ would be discarded, because the edge between $o_1$ and $o_2$ would be cut. This is a reasonable approach, because conditioned on $o_1$'s information, $t_2$ is independent and therefore cannot improve the estimate. 

However, on a loopy graph, this approach discards viable message paths, thereby placing undue weight paths that might otherwise have a very low likelihood. For example, consider the graph in Figure \ref{fig:pruning2}. Even though the most likely path for the message is $0\rightarrow o_1\rightarrow 3 \rightarrow o_2$, pruning removes the edge between $o_1$ and node $3$, so the estimator acts as if the message traversed the path $0\rightarrow 4\rightarrow 5 \rightarrow 6 \rightarrow 7 \rightarrow 3 \rightarrow o_2$---a path with comparatively low likelihood. Although this pruning approach is not strictly correct, we will demonstrate that it performs well in practice. Indeed, our simulation results suggest that the high accuracy levels reported in \cite{pinto} over general graphs are mostly due to pruning, not the proposed estimator.
\begin{figure}[h]
\centering
\includegraphics[width= 3in]{figures/pruning2}
\caption{The pruning used in \cite{pinto} can lead to false likelihood computations over loopy graphs; the scheme discounts the possibility of a spy passing the message to another spy.
%Delays $\theta_{ij}$ are modeled as Gaussians $\mathcal N(2,0.5)$, and spreading was run for 8 time units.
}
\label{fig:pruning2}
%\vspace*{-0.4in}
\end{figure}

\subsubsection{Trees}
We first tested the estimator on trees. As assumed in \cite{pinto}, each node transmits the message to its neighbors with an iid delay that is distributed according to $\mathcal N(2,0.5)$. Figure \ref{fig:pd_vs_spies} shows the probability of detection as a function of the fraction of spies over a 3-regular tree. We ran the simulation for 8 timesteps, leading to an average graph size of 75 nodes. Each datapoint is averaged over 4500 trials. Although \cite{pinto} does not provide simulation results over trees, we observe high detection probabilities for relatively low fractions of spies; with only 5 percent of the nodes spying, the source gets caught with probability 0.3. Note also that the ML estimator performs equally well with or without pruning (the plot lines are jittered to show both); this is because over trees, pruning never removes feasible candidates nodes or edges. We will see later that pruning over loopy graphs can remove feasible edges, and thereby impact the probability of detection. 
For reference, Figure \ref{fig:pd_vs_spies} also shows the probability of detection by the first-spy estimator. The ML estimator clearly outperforms the first-spy estimator, which is a good sanity check. Moreover, the first-spy estimator with pruning closely follows the theoretically-expected probability of detection; for a fraction of spies $p$, this probability grows as $P(\hat v=v^*)=1-(1-p)^d$ where $d$ denotes the degree of the underlying tree.
Figure \ref{fig:hops_vs_spies} shows the corresponding hop distances between the estimated source and the true source. We observe that with as few as 30 percent spies, this average hop distance is less than 0.1; the diameter of the graph is 8 hops. 
\begin{figure}
\centering
\includegraphics[height = 2.4in]{figures/pd_vs_spies}
\caption{Probability of detection, i.e. $P(\hat v = v^*)$, as a function of the spy probability $p$. This plot was generated over 3-regular trees. %Delays $\theta_{ij}$ are modeled as Gaussians $\mathcal N(2,0.5)$, and spreading was run for 8 time units.
}
\label{fig:pd_vs_spies}
%\vspace*{-0.4in}
\end{figure}

\begin{figure}
\centering
\includegraphics[height = 2.4in]{figures/hops_vs_spies}
\caption{Hop distance of the estimate $\hat v$ from the true source $v^*$ as a function of the spy probability $p$. This plot was generated over 3-regular trees. %Delays $\theta_{ij}$ are modeled as Gaussians $\mathcal N(2,0.5)$, and spreading was run for 8 time units.
}
\label{fig:hops_vs_spies}
%\vspace*{-0.4in}
\end{figure}

\subsection{Barabasi-Albert graphs}
We next considered random graphs with loops. \cite{pinto} considers a number of random graph structures, including Apollonian, Erdos-Renyi, and Barabasi-Albert. Due to time constraints, we only considered the latter. However, as mentioned earlier, we did not have access to the parameters used to generate the graphs in \cite{pinto}. To verify that we computed the estimator correctly, we considered a range of different parameters, and show that our results are at least plausibly consistent with the results provided in \cite{pinto}.

Figures \ref{fig:ba_graph} and \ref{fig:ba_graph_p1} show the probability of detection as a function of the fraction of spies for Barabasi-Albert graphs with $N=100$ nodes. We considered a minimum degree $m$ of 1 and 5. Each datapoint is averaged over 2000 trials. The figures illustrate that our numbers are at least plausible given the measurement provided in \cite{pinto}, albeit slightly low. We believe that over graphs with a higher minimum degree (e.g., $m=6$), the optimal estimator could reach 90 percent accuracy with 41 percent spies. Given the estimator's strong performance on trees, we believe that our estimator implementation is consistent with that in \cite{pinto}. %It also gives the right numbers when we compute the likelihoods for small graphs by hand. 

Note that the first-spy estimator with pruning performs as well or better than the `ML' estimator (which is not actually ML over general graph structures). Moreover, by plotting the pruned graph $\mathcal G_a$, we observed that $\mathcal G_a$ generally has very few honest nodes (about 10) when the fraction of spies is about 40; the pruned graph size decreases with spy fraction. 
We want to highlight that the first-spy estimator with pruning always returns a node from $\mathcal G_a$ by construction.  Moreover, the first-spy estimator achieves comparable accuracy to the ML estimator with orders of magnitude less computation; the first-spy estimator has complexity $O(1)$ as compared to $O(N^3)$ for the ML estimator. As such, one of the main takeaways from our study is that when direction-of-infection information is available and the graph not tree-structured, there is no reason to use the estimator from \cite{pinto}; the first-spy estimator will do as well or better, while using only a fraction of the computation. 

Conversely, when there is no direction-of-infection information, the first-spy estimator performs quite poorly compared the estimator from \cite{pinto}. Even the hop distances, illustrated in Figures \ref{fig:ba_hops} and \ref{fig:ba_hops_p1}, are significantly higher than the other tested estimators for the first-spy estimator without pruning. As such, the Pinto \emph{et al.} estimator may be quite powerful when there is no direction information available; this problem setting was not explored in their original paper.

\begin{figure*}[ht] \label{ fig7} 
  \begin{minipage}[b]{0.48\linewidth}
    \includegraphics[width=3in]{figures/ba_graphs} 
    \caption{Probability of detection vs. spy fraction over Barabasi-Albert graphs with minimum degree $m=5$ and number of nodes $N=100$.} 
\label{fig:ba_graph}
  \end{minipage} 
\begin{minipage}[b]{0.48\linewidth}
    \includegraphics[width=3in]{figures/ba_graphs_p1} 
    \caption{Probability of detection vs. fraction over Barabasi-Albert graphs with minimum degree $m=1$ and number of nodes $N=100$.} 
\label{fig:ba_graph_p1}
  \end{minipage}
  \begin{minipage}[b]{0.48\linewidth}
    \includegraphics[width=3in]{figures/ba_hops} 
    \caption{Hop distance from estimate to true source vs. spy fraction over Barabasi-Albert graphs with minimum degree $m=5$ and number of nodes $N=100$.} 
\label{fig:ba_hops}
  \end{minipage} 
  \hfill
  \begin{minipage}[b]{0.48\linewidth}
    \includegraphics[width=3in]{figures/ba_hops_p1} 
    \caption{Hop distance from estimate to true source vs. spy fraction over Barabasi-Albert graphs with minimum degree $m=1$ and number of nodes $N=100$.} 
  \end{minipage} 
\label{fig:ba_hops_p1}
\end{figure*}


\subsection{Facebook dataset}

Randomized graphs provide decent estimates for the estimators' effectivness, but they are still different from real datasets. In order to test our estimators on realistic data, we use a Facebook dataset~\cite{viswanath-2009-activity} from New Orleans. The dataset contains all of the user-to-user connections on Facebook in New Orleans. Since the number of users is large, we take the first 800 nodes in the graph (the dataset orders the links) and construct a connected subgraph. We then use this subgraph to run our experiments by randomly choosing a propagation source. Because of limited time, we ran 30 trials for both 5\% and 15\% spy ratio.

Table~\ref{table:fb:05} and table~\ref{table:fb:15} show the estimators' results for 30 trials with 5\% and 15\% randomly corrupted nodes. 

\begin{figure}
\label{table:fb:05}
\begin{tabular}{c | c | c}
  & With direction & W/o direction \\
  \hline
  First spy & 30.00 \% & 0.00 \% \\ 
  Max Likelihood & 40.00 \% & 16.67 \% \\
\end{tabular}
\caption{Estimators' performance (percentage of accurate for 30 trials) on the Facebook dataset: 5\% random spies}
\end{figure}


\begin{figure}
\label{table:fb:15}
\begin{tabular}{c | c | c}
  & With direction & W/o direction \\
  \hline
  First spy & 66.67 \% & 3.33 \% \\ 
  Max Likelihood & 63.33\% & 26.67 \% \\
\end{tabular}
\caption{Estimators' performance (percentage of accurate for 30 trials) on the Facebook dataset: 15\% random spies}
\end{figure}

\section{Future work}

\todo{todo}


% A category with the (minimum) three required fields
%\category{H.4}{Information Systems Applications}{Miscellaneous}
%A category including the fourth, optional field follows...
%\category{D.2.8}{Software Engineering}{Metrics}[complexity measures,
%performance measures]


%\terms{Systems}

%\keywords{ACM proceedings, \LaTeX, text tagging}





%
% The following two commands are all you need in the
% initial runs of your .tex file to
% produce the bibliography for the citations in your paper.
%% \bibliographystyle{abbrv}
%%     \bibliography{related_work,rp,rp-str,rp-conf}
% sigproc.bib is the name of the Bibliography in this case
% You must have a proper ``.bib'' file
%  and remember to run:
% latex bibtex latex latex
% to resolve all references

\bibliography{references}
\bibliographystyle{ieeetr}

%
% ACM needs 'a single self-contained file'!
%
%APPENDICES are optional
\balancecolumns

% That's all folks!
\end{document}

}
