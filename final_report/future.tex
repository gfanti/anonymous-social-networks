\section{Future work}

Due to some major roadblocks in our project (mainly problems with the observer model estimator), we were not able to accomplish some of our original goals. 

\begin{itemize}

\item In the simulation, all nodes choose to ``like'' a message (unless it is a message that it has seen before). This is the best case scenario because the spy nodes are able to obtain the most amount of information. Thus, our simulation provides an upper bound on the source detection accuracy. The future work here would be to adjust the message propagation likelihood. The easiest parameter here is to change is for an honest node to \emph{probabilistically} like a new message: when an honest node $n$ receives a new message $M$, it ``likes'' $M$ with some probability $p$, and drops $M$ with probability $1-p$.

\item Spy node distribution is another parameter we would have liked to adjust. In this paper, we used uniformly random distribution for spy nodes. This might not be a realistic estimate and in reality, one would expect some bias in the spy node sampling. For example, a political candidate may only be able to corrupt nodes that are closely related to him/her in the social network, therefore reducing the effectiveness of source detection. Another approch we can try here is similar to the one described in the paper: corrupting high degree nodes. Intuitively, this should give better accuracy because 1. with direction information, this node will get more accurate information 2. under the probabilistic propagation model, the node is more likely to receive messages. This is not such a realistic node because popular accounts are less likely to be hacked, but it will be an interesting case for comparison nonetheless.

\end{itemize}
